
%% bare_jrnl.tex
%% V1.4b
%% 2015/08/26
%% by Michael Shell
%% see http://www.michaelshell.org/
%% for current contact information.
%%
%% This is a skeleton file demonstrating the use of IEEEtran.cls
%% (requires IEEEtran.cls version 1.8b or later) with an IEEE
%% journal paper.
%%
%% Support sites:
%% http://www.michaelshell.org/tex/ieeetran/
%% http://www.ctan.org/pkg/ieeetran
%% and
%% http://www.ieee.org/

%%*************************************************************************
%% Legal Notice:
%% This code is offered as-is without any warranty either expressed or
%% implied; without even the implied warranty of MERCHANTABILITY or
%% FITNESS FOR A PARTICULAR PURPOSE! 
%% User assumes all risk.
%% In no event shall the IEEE or any contributor to this code be liable for
%% any damages or losses, including, but not limited to, incidental,
%% consequential, or any other damages, resulting from the use or misuse
%% of any information contained here.
%%
%% All comments are the opinions of their respective authors and are not
%% necessarily endorsed by the IEEE.
%%
%% This work is distributed under the LaTeX Project Public License (LPPL)
%% ( http://www.latex-project.org/ ) version 1.3, and may be freely used,
%% distributed and modified. A copy of the LPPL, version 1.3, is included
%% in the base LaTeX documentation of all distributions of LaTeX released
%% 2003/12/01 or later.
%% Retain all contribution notices and credits.
%% ** Modified files should be clearly indicated as such, including  **
%% ** renaming them and changing author support contact information. **
%%*************************************************************************


% *** Authors should verify (and, if needed, correct) their LaTeX system  ***
% *** with the testflow diagnostic prior to trusting their LaTeX platform ***
% *** with production work. The IEEE's font choices and paper sizes can   ***
% *** trigger bugs that do not appear when using other class files.       ***                          ***
% The testflow support page is at:
% http://www.michaelshell.org/tex/testflow/



\documentclass[journal]{IEEEtran}


% correct bad hyphenation here
\hyphenation{op-tical net-works semi-conduc-tor}


\begin{document}

\title{INF6600 Rapport TP4 \\ Améliorations au système de contrôle d'un drone fermier}


\author{Daniel~Lussier-Lévesque et Ian~Gagnon}

% make the title area
\maketitle

% As a general rule, do not put math, special symbols or citations
% in the abstract or keywords.
\begin{abstract}
L'implémentation d'un système de contrôle de drone fermier a été réalisé progressivement tout au long de la session en commençant par un module TrueTime avec un système continu déjà implémenté en Simulink. Par la suite, l'implémentation a continué sur une machine virtuelle QNX roulant sous Vmware, avec certaines fonctions déjà fournies. Ce rapport focus sur ce qui a été fait sur cette base de code pour améliorer les aspects les plus importants. La première étape a été d'implémenter un système de log qui nous permettent de voir comment se comporte l'ordonnancement des tâches et la performance du système. Par la suite, malgré le manque de droits d'accès qui nous permettraient de comprendre l'ensemble des problèmes de performances, nous avons toutefois été capable de les mesurer et de les documenter. Un système graphique vivant en dehors de la machine virtuelle nous permettant de valider le comportement du système a aussi été implémenté.
\end{abstract}


\section{Introduction}
\IEEEPARstart{L}{e} système de contrôle considéré est celui d'un drone autonome capable de se déplacer sur trois axes et ayant une caméra fixe pour prendre en photo l'ensemble d'un champs. Le drone doit naviguer les champs et prendre des photos jusqu'à ce que la mémoire soit pleine, puis transmettre les photos à une station base. Lorsque la batterie est presque vide (10\%), le drone doit retourner à la station base pour être rechargé. 

Le système de contrôle implémente le contrôle de la navigation (où le drone va), le contrôle de la batterie (est-il temps d'aller recharger la batterie?) et le contrôle mission (quelle séquence d'étapes doit être exécutés pour opérer avec succès?). Nous avons choisi de mettre è jour le système de contrôle du drone toutes les 100 ms.

De plus, l'implémentation du système de contrôle doit être accompagné d'une plateforme de simulation d'un environnement dans lequel opérer, qui comprend le temps de réponse de la caméra, les variations d'orientation et de vitesses, le temps de transmission de photo à la station base et la charge et décharge de la batterie. Le système continu est mis à jour toutes les 20ms de sorte qu'il soit significativement plus rapide que le système de contrôle. De cette manière 

Le système de contrôle de drone sur lequel nous avons apportés nos modifications est un système simple où la communication entre le système continu et discret est effectué par des variables atomiques et par l'enregistrement de \textit{callbacks}. La communication en dehors du système s'effectue par des messages de log sur la console de debug où à intervalle régulier (chaque seconde) l'état du drone est envoyé.

La politique d'ordonnancement utilisée est un systèmes de priorités fixes et un ordonnancement FIFO pour chaque niveau de priorité. Le choix a été fait de placer la priorité du contrôle caméra avant celle du contrôleur de navigation parce qu'il est plus facile de compenser pour une échéance manquée dans le cas de la navigation (une photo manquée implique de défaire les dernières mise-à-jour de navigation pour reprendre la photo).


% An example of a floating figure using the graphicx package.
% Note that \label must occur AFTER (or within) \caption.
% For figures, \caption should occur after the \includegraphics.
% Note that IEEEtran v1.7 and later has special internal code that
% is designed to preserve the operation of \label within \caption
% even when the captionsoff option is in effect. However, because
% of issues like this, it may be the safest practice to put all your
% \label just after \caption rather than within \caption{}.
%
% Reminder: the "draftcls" or "draftclsnofoot", not "draft", class
% option should be used if it is desired that the figures are to be
% displayed while in draft mode.
%
%\begin{figure}[!t]
%\centering
%\includegraphics[width=2.5in]{myfigure}
% where an .eps filename suffix will be assumed under latex, 
% and a .pdf suffix will be assumed for pdflatex; or what has been declared
% via \DeclareGraphicsExtensions.
%\caption{Simulation results for the network.}
%\label{fig_sim}
%\end{figure}

% Note that the IEEE typically puts floats only at the top, even when this
% results in a large percentage of a column being occupied by floats.


% An example of a double column floating figure using two subfigures.
% (The subfig.sty package must be loaded for this to work.)
% The subfigure \label commands are set within each subfloat command,
% and the \label for the overall figure must come after \caption.
% \hfil is used as a separator to get equal spacing.
% Watch out that the combined width of all the subfigures on a 
% line do not exceed the text width or a line break will occur.
%
%\begin{figure*}[!t]
%\centering
%\subfloat[Case I]{\includegraphics[width=2.5in]{box}%
%\label{fig_first_case}}
%\hfil
%\subfloat[Case II]{\includegraphics[width=2.5in]{box}%
%\label{fig_second_case}}
%\caption{Simulation results for the network.}
%\label{fig_sim}
%\end{figure*}


\section{Lacunes du système de base}

\section{Améliorations apportées}

\section{Résultats}


\section{Conclusion}
The conclusion goes here.



% if have a single appendix:
%\appendix[Proof of the Zonklar Equations]
% or
%\appendix  % for no appendix heading
% do not use \section anymore after \appendix, only \section*
% is possibly needed

% use appendices with more than one appendix
% then use \section to start each appendix
% you must declare a \section before using any
% \subsection or using \label (\appendices by itself
% starts a section numbered zero.)
%

% references section

% can use a bibliography generated by BibTeX as a .bbl file
% BibTeX documentation can be easily obtained at:
% http://mirror.ctan.org/biblio/bibtex/contrib/doc/
% The IEEEtran BibTeX style support page is at:
% http://www.michaelshell.org/tex/ieeetran/bibtex/
%\bibliographystyle{IEEEtran}
% argument is your BibTeX string definitions and bibliography database(s)
%\bibliography{IEEEabrv,../bib/paper}
%
% <OR> manually copy in the resultant .bbl file
% set second argument of \begin to the number of references
% (used to reserve space for the reference number labels box)
\begin{thebibliography}{1}

\bibitem{IEEEhowto:kopka}
H.~Kopka and P.~W. Daly, \emph{A Guide to \LaTeX}, 3rd~ed.\hskip 1em plus
  0.5em minus 0.4em\relax Harlow, England: Addison-Wesley, 1999.

\end{thebibliography}




% Can be used to pull up biographies so that the bottom of the last one
% is flush with the other column.
%\enlargethispage{-5in}



% that's all folks
\end{document}


